\documentclass[a4paper,12px]{article}
\usepackage{graphicx}
\usepackage[english]{babel}
\usepackage{fancyhdr}
\usepackage{lastpage}
\usepackage{xifthen}
\usepackage[linesnumberedhidden, titlenotnumbered]{algorithm2e}
\usepackage{lipsum}
\usepackage{hyperref}
\usepackage{array}
\usepackage{tabularx}
\usepackage{caption}
\usepackage{amsfonts}
\usepackage{amssymb}
\usepackage{amsmath}
\usepackage{placeins}
\usepackage{enumitem}

\usepackage{minted}
\usepackage{listings}
\usepackage{dsfont}
\usepackage{units}

\pagestyle{fancy}
\lhead{\includegraphics[width=7cm]{logoUvA}}
\rhead{\footnotesize \textsc {Report\\ \opdracht}}
\lfoot
{%
    \footnotesize \studentA
    \ifthenelse{\isundefined{\studentB}}{}{\\ \studentB}
    \ifthenelse{\isundefined{\studentC}}{}{\\ \studentC}
    \ifthenelse{\isundefined{\studentD}}{}{\\ \studentD}
    \ifthenelse{\isundefined{\studentE}}{}{\\ \studentE}
}
\cfoot{}
\rfoot{\small \textsc {Page \thepage\ of \pageref{LastPage}}}
\renewcommand{\footrulewidth}{0.5pt}

\fancypagestyle{firststyle}
{%
    \fancyhf{}
    \renewcommand{\headrulewidth}{0pt}
    \chead{\includegraphics[width=7cm]{logoUvA}}
    \rfoot{\small \textsc {Page \thepage\ of \pageref{LastPage}}}
}

\setlength{\topmargin}{-0.3in}
\setlength{\textheight}{630pt}
\setlength{\headsep}{40pt}
\setlength{\parindent}{0pt}

% =================================== DOC INFO ===================================

\newcommand{\opdracht}{Statistisch Redeneren}
\newcommand{\titel}{Lab 2}
\newcommand{\docent}{Rein van de Boomgaard}
\newcommand{\cursus}{Statistisch Redeneren}
\newcommand{\vakcode}{5062STRE6Y}
\newcommand{\datum}{\today}
\newcommand{\studentA}{Maico Timmerman}
\newcommand{\uvanetidA}{10542590}
\newcommand{\studentB}{Tim van Zalingen}
\newcommand{\uvanetidB}{10784012}
% \newcommand{\studentC}{Boudewijn Braams}
\newcommand{\uvanetidC}{10401040}
% \newcommand{\studentD}{Govert Verkes}
\newcommand{\uvanetidD}{10211748}
%\newcommand{\studentE}{Naam student 5}
\newcommand{\uvanetidE}{UvAnetID student 5}

% ===================================  ===================================

\begin{document}
\thispagestyle{firststyle}
\begin{center}
    \textsc{\Large \opdracht}\\[0.2cm]
    \rule{\linewidth}{0.5pt} \\[0.4cm]
    {\huge \bfseries \titel}
    \rule{\linewidth}{0.5pt} \\[0.2cm]
    {\large \datum  \\[0.4cm]}

    \begin{minipage}{0.4\textwidth}
        \begin{flushleft}

            \emph{Students:}\\
            {\studentA \\ {\small \uvanetidA \\[0.2cm]}}
            \ifthenelse{\isundefined{\studentB}}{}{\studentB \\ {\small \uvanetidB \\[0.2cm]}}
        \end{flushleft}
    \end{minipage}
    ~%
    \begin{minipage}{0.4\textwidth}
        \begin{flushright}
            \emph{Lecturer:} \\
            \docent \\[0.2cm]
            \emph{Course:} \\
            \cursus \\[0.2cm]
            % \emph{Student:}\\
            \ifthenelse{\isundefined{\studentC}}{}{\studentC \\ {\small \uvanetidC \\[0.2cm]}}
            \ifthenelse{\isundefined{\studentD}}{}{\studentD \\ {\small \uvanetidD \\[0.2cm]}}
            \ifthenelse{\isundefined{\studentE}}{}{\studentE \\ {\small \uvanetidE \\ [0.2cm]}}
        \end{flushright}
    \end{minipage}\\[1 cm]
\end{center}


% =================================== CONTENTS ===================================

\tableofcontents
\clearpage

% =================================== MAIN TEXT ===================================


\section{Kansrekening 2}
\subsection{Opgave 1}
    %Beschouw de uniforme verdeling op interval $[3,9]$.
    \begin{enumerate}[label=(\alph*)]
        \item De lineaire functie $F(x)$ stijgt op dit interval van 0 naar 1. Dit geeft ons:
        \begin{equation}
            F(x)=\dfrac{x-3}{9-3}=\dfrac{x-3}{6}
        \end{equation}
        Voor $2\leq x \leq 8$.
        \item
            \begin{equation}
                P([-10,3])=F(-10)-F(3)=0-\dfrac{3-3}{6}=0
            \end{equation}
        \item
            \begin{equation}
                P([a,b])=F(a)-F(b)=\dfrac{a-3}{6}-\dfrac{b-3}{6}=\dfrac{a-3-b+3}{6}=\dfrac{a-b}{6}
            \end{equation}
    \end{enumerate}
\subsection{Opgave 2}
    \begin{enumerate}[label=(\alph*)]
        \item $U=\{\text{'kop'},\text{'munt'}\}$
        \item
            \begin{equation}
                P(k)=\binom{n}{k}p^k(1-p)^{n-k}
            \end{equation}
        \item Dit is de binomiale verdeling, met paramters $n$ en $p$.
        \item
            \inputminted{python}{2d.py}
    \end{enumerate}


%TODO More discussion?

% =================================== REFERENCES ===================================

%\clearpage
% \bibliographystyle{apalike}
% \bibliography{report}

\end{document}
