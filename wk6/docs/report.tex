\documentclass[a4paper,12px]{article}
\usepackage{graphicx}
\usepackage[english]{babel}
\usepackage{fancyhdr}
\usepackage{lastpage}
\usepackage{xifthen}
\usepackage[linesnumberedhidden, titlenotnumbered]{algorithm2e}
\usepackage{lipsum}
\usepackage{hyperref}
\usepackage{array}
\usepackage{tabularx}
\usepackage{caption}
\usepackage{amsfonts}
\usepackage{amssymb}
\usepackage{amsmath}
\usepackage{placeins}
\usepackage{enumitem}

\usepackage{parskip}
\usepackage{minted}
\usepackage{listings}
\usepackage{dsfont}
\usepackage{units}

\pagestyle{fancy}
\lhead{\includegraphics[width=7cm]{logoUvA}}
\rhead{\footnotesize \textsc {Report\\ \opdracht}}
\lfoot
{%
    \footnotesize \studentA
    \ifthenelse{\isundefined{\studentB}}{}{\\ \studentB}
    \ifthenelse{\isundefined{\studentC}}{}{\\ \studentC}
    \ifthenelse{\isundefined{\studentD}}{}{\\ \studentD}
    \ifthenelse{\isundefined{\studentE}}{}{\\ \studentE}
}
\cfoot{}
\rfoot{\small \textsc {Page \thepage\ of \pageref{LastPage}}}
\renewcommand{\footrulewidth}{0.5pt}

\fancypagestyle{firststyle}
{%
    \fancyhf{}
    \renewcommand{\headrulewidth}{0pt}
    \chead{\includegraphics[width=7cm]{logoUvA}}
    \rfoot{\small \textsc {Page \thepage\ of \pageref{LastPage}}}
}

\setlength{\topmargin}{-0.3in}
\setlength{\textheight}{630pt}
\setlength{\headsep}{40pt}

% =================================== DOC INFO ===================================

\newcommand{\opdracht}{Statistisch Redeneren}
\newcommand{\titel}{Lab 6}
\newcommand{\docent}{Rein van de Boomgaard}
\newcommand{\cursus}{Statistisch Redeneren}
\newcommand{\vakcode}{5062STRE6Y}
\newcommand{\datum}{\today}
\newcommand{\studentA}{Maico Timmerman}
\newcommand{\uvanetidA}{10542590}
\newcommand{\studentB}{Tim van Zalingen}
\newcommand{\uvanetidB}{10784012}
% \newcommand{\studentC}{Boudewijn Braams}
\newcommand{\uvanetidC}{10401040}
% \newcommand{\studentD}{Govert Verkes}
\newcommand{\uvanetidD}{10211748}
%\newcommand{\studentE}{Naam student 5}
\newcommand{\uvanetidE}{UvAnetID student 5}

% ===================================  ===================================

\begin{document}
\thispagestyle{firststyle}
\begin{center}
    \textsc{\Large \opdracht}\\[0.2cm]
    \rule{\linewidth}{0.5pt} \\[0.4cm]
    {\huge \bfseries \titel}
    \rule{\linewidth}{0.5pt} \\[0.2cm]
    {\large \datum  \\[0.4cm]}

    \begin{minipage}{0.4\textwidth}
        \begin{flushleft}

            \emph{Students:}\\
            {\studentA \\ {\small \uvanetidA \\[0.2cm]}}
            \ifthenelse{\isundefined{\studentB}}{}{\studentB \\ {\small \uvanetidB \\[0.2cm]}}
        \end{flushleft}
    \end{minipage}
    ~%
    \begin{minipage}{0.4\textwidth}
        \begin{flushright}
            \emph{Lecturer:} \\
            \docent \\[0.2cm]
            \emph{Course:} \\
            \cursus \\[0.2cm]
            % \emph{Student:}\\
            \ifthenelse{\isundefined{\studentC}}{}{\studentC \\ {\small \uvanetidC \\[0.2cm]}}
            \ifthenelse{\isundefined{\studentD}}{}{\studentD \\ {\small \uvanetidD \\[0.2cm]}}
            \ifthenelse{\isundefined{\studentE}}{}{\studentE \\ {\small \uvanetidE \\ [0.2cm]}}
        \end{flushright}
    \end{minipage}\\[1 cm]
\end{center}


% =================================== CONTENTS ===================================

\tableofcontents
\clearpage

% =================================== MAIN TEXT ===================================

\section{Support vector Machine Classification}

%In de lecture notes (handout) staan opgaven over de SVM. Deze opgaven gaan uit
%van het gebruik van libsvm. Deze library is een defacto standaard geworden voor
%het gebruik van support vector machines. Het is ook mogelijk om gebruik te
%maken van sklearn (zie http://scikit-learn.org). Dit lijkt (nog niet zelf veel
%gebruikt) een fraaie verzameling van machine learning tools. Waaronder ook SVM
%(en waarschijnlijk is dat gebaseerd op libsvm...) Je mag voor deze opdracht
%zelf een keuze maken.

%Opgave 4.7 gebruikt een SVM om de spectra uit een eerdere opdracht te
%classificeren als een kleur ('rood', 'groen', etc).
%
%Zowel bij gebruik van libsvm als van sklearn is het belangrijk dat je niet zelf
%de waarde van C en eventuele kernel parameters kiest maar deze met een bruto
%force search bepaald (in sklearn heet dat netjes een grid search en is er zelfs
%een speciale functie die dat voor je doet).
%
%In het verslag alleen het classificeren van de spectra beschrijven. Daarin moet
%komen:
%
% - Korte beschrijving van SVM classificatie
% - Beschrijving van de data
%
% - Gevolgde procedure (gridsearch + leren + testen), geef daarbij duidelijk aan
%   hoe je met de set van voorbeelden bent om gegaan.
%
% - Beschrijf de resultaten (percentages en voorbeelden van wanneer het fout
%   gaat)
%

Support Vector Machines (SVM) can be used to classify data in separate classes.
An SVM is a linear classifier not based on probability but on separation.  If
this linear separation cannot be done, a `kernel trick' is used. The data is
transformed into a higher dimension, which makes it possible for linear
separation. The SVM tries to find a linear separation with maximum margin. To
separate multiple classes, multiple SVMs are used, from which the output is
compared.\\

The data of the color dataset is based on spectra values. A vector of values is
generated for all colors. These high dimensional points, need to be separated in
all different colors. From the dataset, 80\% of the data is used to train the
SVM\@. The data from the color dataset is not linear separable, therefor the
data is transformed using the kernel trick. To find the right parameters for
this kernel trick, a grid search is performed over all the parameters.\\

After running the gridsearch optimized SVM we reached an accuracy of 0.878049
for 4141 tests (3636 correct). SKlearn was complaining that the smallest class
with one sample could not be classified. Removing the sample from the dataset
made no difference in the accuracy.

% =================================== REFERENCES ===================================

%\clearpage
% \bibliographystyle{apalike}
% \bibliography{report}

\end{document}
